\documentclass[paper=a4]{scrartcl}

\usepackage[T1]{fontenc}
\usepackage[ngerman]{babel}
\usepackage[utf8]{inputenc}
\usepackage{listings}

\usepackage{lmodern}
\usepackage{listings}
\usepackage{xcolor}


\lstdefinelanguage{myPOVlanguage}
{
  alsoletter  = {\#},
  keywords    =
  {
    \#include, \#declare, \#version,
    assumed_gamma,
    background, box,
    camera, color, color_map, cone, cylinder,
    direction,
    function,
    global_settings, gradient,
    light_source, location, look_at,
    matrix,
    no_shadow,
    object,
    pigment, pow,
    rgb, right, rotate,
    scale, sphere,
    translate,
    x, y, z,
  },
  sensitive   = true,
  string      = [b]{"},
  comment     = [l]{//},
  morecomment = [s]{/*}{*/},
}

% --- color definitions ---
\definecolor{povcodered}{rgb}{0.75,0.25,0.25}
\definecolor{povcodegreen}{rgb}{0.25,0.75,0.25}
\definecolor{povcodeblue}{rgb}{0.25,0.25,0.75}
\definecolor{povcodepurple}{rgb}{0.5,0,0.35}
\definecolor{povcodebluegreen}{rgb}{0,0.5,0.5}

\lstdefinestyle{myPOVstyle}
{
  language           = myPOVlanguage,
  frame              = single,
  framextopmargin    = 3mm,
  framexbottommargin = 3mm,
  basicstyle         = \ttfamily\bfseries,
  keywordstyle       = \color{povcodepurple},
  stringstyle        = \color{povcodered},
  commentstyle       = \color{povcodegreen}\itshape,
  literate =*
  {0}{{{\color{povcodebluegreen}0}}}1
  {1}{{{\color{povcodebluegreen}1}}}1
  {2}{{{\color{povcodebluegreen}2}}}1
  {3}{{{\color{povcodebluegreen}3}}}1
  {4}{{{\color{povcodebluegreen}4}}}1
  {5}{{{\color{povcodebluegreen}5}}}1
  {6}{{{\color{povcodebluegreen}6}}}1
  {7}{{{\color{povcodebluegreen}7}}}1
  {8}{{{\color{povcodebluegreen}8}}}1
  {9}{{{\color{povcodebluegreen}9}}}1
  {+}{{{\color{povcodered}+}}}1
  {-}{{{\color{povcodered}-}}}1
  {*}{{{\color{povcodered}*}}}1
  {<}{{{\color{povcodered}<}}}1
  {>}{{{\color{povcodered}>}}}1
  {\{}{{{\color{povcodeblue}\{}}}1
  {\}}{{{\color{povcodeblue}\}}}}1
}

% --- patch to get proper highlighting of / and // ---
\makeatletter
\lst@AddToHook{OutputOther}
{%
  \edef\@tempa{\the\lst@token\relax}%
  %
  % apply \color{povcodered} if / is found
  \expandafter\expandafter\expandafter\ifx\expandafter\@firstoftwo\@tempa/%
  \def\lst@thestyle{\color{povcodered}}%
  %
  % apply comment style if // is found
  \expandafter\expandafter\expandafter\ifx\expandafter\@secondoftwo\@tempa/%
  \def\lst@thestyle{\lst@commentstyle}%
  \fi
  \fi
}
\makeatother

\lstset{style=myPOVstyle}




\begin{document}

\title{Doku Gruppe 42 - IVC WiSe 15/16}
\author{Kamila Ignatowicz\\ Marco Pfomann \\ Felix Favre}
\date{\today}
\maketitle

\section{Umgebung}
\subsection{Himmel}
Für den Sternenhimmel wurde eine sky\_shpere mit bozo pigment genutzt
\begin{lstlisting}
sky_sphere {
  pigment {
    bozo
    color_map {
      [0.0 White*3]
      [0.2 Black]
      [1.0 Black]
    }
    scale .006
  }
}
\end{lstlisting}

\subsection{Boden}
Der Boden ist eine einfache graue Plane auf der xz-Ebene
\begin{lstlisting}
plane { y, 0 pigment { color red 0.1 green 0.1 blue 0.1} }
\end{lstlisting}

\subsection{Media}
Damit die Lichtstrahlen der Laser und Movindheads in der Luft zu sehen sind ist ein media nötig. Hier wurde bewusst kein atmospheric media genutzt, da povray nicht in der Lage ist unendlich lange Lichtstralen in einem media zu samplen.
Für unsere Zwecke Reicht eine Box mit 200x152x160 Metern.
Der extinction Wert ist so gering gesetzt, weil wir mit Distanzen von über 100 Metern arbeiten.
\begin{lstlisting}
box{<-50,-2,10>,<150,150,-150> pigment{rgbt 1}
  interior{
    media {
      scattering {
        4,
        1
        extinction 0.001
      }
      method 3
    }
  }
  hollow
}
\end{lstlisting}

\subsection{Includes}
Der Rest der Szene wurde in eigenen Dateien definiert und per \#include Statement eingebunden

\begin{lstlisting}
#include "helpers.inc"
#include "lightcolors.inc"
#include "switches.inc"

#include "camera.pov"
#include "lasers.pov"
#include "buehne.pov"
#include "movingheads_buehne.pov"
#include "movingheads_tower.pov"
#include "firework.pov"
#include "dj.pov"
#include "fans.pov"
\end{lstlisting}

\section{Kamera}
TODO: Marco
\section{Buehne}
TODO: Felix
\section{Laser und Movingheads}
TODO: Felix
\section{Fans und DJ}
TODO. Marco
\section{DJ-Pult}
TODO: Felix + Marco


\end{document}
