\documentclass[paper=a4]{scrartcl}

\usepackage[T1]{fontenc}
\usepackage[ngerman]{babel}
\usepackage[utf8]{inputenc}
\usepackage{listings}

\usepackage{lmodern}
\usepackage{listings}
\usepackage{xcolor}


\lstdefinelanguage{myPOVlanguage}
{
  alsoletter  = {\#},
  keywords    =
  {
    \#include, \#declare, \#version,
    assumed_gamma,
    background, box,
    camera, color, color_map, cone, cylinder,
    direction,
    function,
    global_settings, gradient,
    light_source, location, look_at,
    matrix,
    no_shadow,
    object,
    pigment, pow,
    rgb, right, rotate,
    scale, sphere,
    translate,
    x, y, z,
  },
  sensitive   = true,
  string      = [b]{"},
  comment     = [l]{//},
  morecomment = [s]{/*}{*/},
}

% --- color definitions ---
\definecolor{povcodered}{rgb}{0.75,0.25,0.25}
\definecolor{povcodegreen}{rgb}{0.25,0.75,0.25}
\definecolor{povcodeblue}{rgb}{0.25,0.25,0.75}
\definecolor{povcodepurple}{rgb}{0.5,0,0.35}
\definecolor{povcodebluegreen}{rgb}{0,0.5,0.5}

\lstdefinestyle{myPOVstyle}
{
  language           = myPOVlanguage,
  frame              = single,
  framextopmargin    = 3mm,
  framexbottommargin = 3mm,
  basicstyle         = \ttfamily\bfseries,
  keywordstyle       = \color{povcodepurple},
  stringstyle        = \color{povcodered},
  commentstyle       = \color{povcodegreen}\itshape,
  literate =*
  {0}{{{\color{povcodebluegreen}0}}}1
  {1}{{{\color{povcodebluegreen}1}}}1
  {2}{{{\color{povcodebluegreen}2}}}1
  {3}{{{\color{povcodebluegreen}3}}}1
  {4}{{{\color{povcodebluegreen}4}}}1
  {5}{{{\color{povcodebluegreen}5}}}1
  {6}{{{\color{povcodebluegreen}6}}}1
  {7}{{{\color{povcodebluegreen}7}}}1
  {8}{{{\color{povcodebluegreen}8}}}1
  {9}{{{\color{povcodebluegreen}9}}}1
  {+}{{{\color{povcodered}+}}}1
  {-}{{{\color{povcodered}-}}}1
  {*}{{{\color{povcodered}*}}}1
  {<}{{{\color{povcodered}<}}}1
  {>}{{{\color{povcodered}>}}}1
  {\{}{{{\color{povcodeblue}\{}}}1
  {\}}{{{\color{povcodeblue}\}}}}1
}

% --- patch to get proper highlighting of / and // ---
\makeatletter
\lst@AddToHook{OutputOther}
{%
  \edef\@tempa{\the\lst@token\relax}%
  %
  % apply \color{povcodered} if / is found
  \expandafter\expandafter\expandafter\ifx\expandafter\@firstoftwo\@tempa/%
  \def\lst@thestyle{\color{povcodered}}%
  %
  % apply comment style if // is found
  \expandafter\expandafter\expandafter\ifx\expandafter\@secondoftwo\@tempa/%
  \def\lst@thestyle{\lst@commentstyle}%
  \fi
  \fi
}
\makeatother

\lstset{style=myPOVstyle}




\begin{document}

\title{Doku Gruppe 42 - IVC WiSe 15/16}
\author{Kamila Ignatowicz\\ Marco Pfomann \\ Felix Favre}
\date{\today}
\maketitle

\section{Umgebung}
\subsection{Himmel}
Für den Sternenhimmel wurde eine sky\_shpere mit bozo pigment genutzt
\begin{lstlisting}
sky_sphere {
  pigment {
    bozo
    color_map {
      [0.0 White*3]
      [0.2 Black]
      [1.0 Black]
    }
    scale .006
  }
}
\end{lstlisting}

\subsection{Boden}
Der Boden ist eine einfache graue Plane auf der xz-Ebene
\begin{lstlisting}
plane { y, 0 pigment { color red 0.1 green 0.1 blue 0.1} }
\end{lstlisting}

\subsection{Media}
Damit die Lichtstrahlen der Laser und Movindheads in der Luft zu sehen sind ist ein media nötig. Hier wurde bewusst kein atmospheric media genutzt, da povray nicht in der Lage ist unendlich lange Lichtstralen in einem media zu samplen.
Für unsere Zwecke Reicht eine Box mit 200x152x160 Metern.
Der extinction Wert ist so gering gesetzt, weil wir mit Distanzen von über 100 Metern arbeiten.
\begin{lstlisting}
box{<-50,-2,10>,<150,150,-150> pigment{rgbt 1}
  interior{
    media {
      scattering {
        4,
        1
        extinction 0.001
      }
      method 3
    }
  }
  hollow
}
\end{lstlisting}

\subsection{Includes}
Der Rest der Szene wurde in eigenen Dateien definiert und per \#include Statement eingebunden

\begin{lstlisting}
#include "helpers.inc"
#include "lightcolors.inc"
#include "switches.inc"

#include "camera.pov"
#include "lasers.pov"
#include "buehne.pov"
#include "movingheads_buehne.pov"
#include "movingheads_tower.pov"
#include "firework.pov"
#include "dj.pov"
#include "fans.pov"
\end{lstlisting}

\section{Kamera}
TODO: Marco
\section{Buehne}
Die Bühe wurde aus einer einfachen Kombination aus box, cylinder und torus Elementen zusammengestellt. Um Codeduplikationen zu vermeiden wurden einzelne Teile als object definiert und mit Rotation und Translation an die Richtige stelle gerückt. Die Farben der einzelnen Elemente wurde in der lightcolors.inc definiert.
Ein Beispiel:
\begin{lstlisting}
#declare towersegment=union{
  box {
    <0, 0, 0>
    <0.5, TowerHeight, 0.7>
  }
  #for (i, 1, 5, 2)
  cylinder {
    <0,0,0>,
    <0,0.01,0>,
    4
    rotate <0,0,90>
    translate <0,i*4.4,5>
  }
  #end
  torus {
    4.15, 0.35
    rotate <0,0,90>
    translate<0.25,4.4,5>
    pigment { color C_TowerBot }
    finish {ambient 1}
  }
  torus {
    4.15,.35
    rotate <0,0,90>
    translate<0.25,3*4.4,5>
    pigment { color C_TowerMid }
    finish {ambient 1}
  }
  torus {
    4.15,0.35
    rotate <0,0,90>
    translate<0.25,5*4.4,5>
    pigment { color C_TowerHi }
    finish {ambient 1}
  }

}

#declare tower=union {
  object { towersegment }
  object { towersegment rotate <0,270,0> translate <10,0,0> }
  object { towersegment rotate <0,180,0> translate <10,0,10> }
  object { towersegment rotate <0,90,0> translate <0,0,10> }
}


\end{lstlisting}
\section{Laser und Movingheads}
Die Laser und Movingheads werden von einer großen Anzahl Spotlights dargestellt.
Für die Laser wurde ein Makro definiert und mithilfe von for-Schleifen vielfach eingesetzt.
Als Parameter hatte dieses die Ursprungsposition des Lasers und die Farbe.
\begin{lstlisting}
#for (i, 0, lasercount, 1)
  laser(<37,2,-0.05>, C_RGBLaser)
#end

\end{lstlisting}
Das Makro wiederum definiert das Spotlight
\begin{lstlisting}
#macro laser (source, lasercolor)
light_source {
  <0,0,0>
  color lasercolor
  spotlight
  radius 0.10
  falloff 0.10
  tightness 50
  media_interaction on
  media_attenuation on
  point_at <0,0,-100>
  rotate <0,(i-(lasercount/2))/2,0>    //Spread
  rotate <0,0,laser_rotate(i)>  //Rotate
  rotate <laser_tilt(i),0,0> //Tilt
  translate source
}
#end
\end{lstlisting}
Die Rotation um x-Achse (Tilt) und z-Achse (Rotate) wurden mit 2 Funktionen berechnet.
Diese haben verschiene Formeln von denen eine Anhand einer Zufallsvariable eine ausgewählt wird. Sowohl die Funktionen selbst als auch der Seed der Zufallsvariable sind von der Clock abhängig. Hiermit wird eine Flüssige Animation erzeugt sowie der Wechsel der Animationen alle paar Sekunden sichergestellt.
\begin{lstlisting}
#local Rand_Lasermode = seed(clock+23);

#declare lasermode_rotate=int(rand(Rand_Lasermode)*3);
#declare lasermode_tilt=int(rand(Rand_Lasermode)*2);

//Only Nice Sine Wave lasers <3
#if (lasermode_tilt = 1)
  #declare lasercount = 50;
  #declare lasermode_rotate = 0;
#end

#local laser_rotate = function(i) {
  #switch (lasermode_rotate)
  #case(0)
    0
    #break
  #case(1)
    360*clock
    #break
  #case(2)
    sin(clock*pi)*45
    #break
  #else
    0
  #end
}

#local laser_tilt = function(i) {
  #switch (lasermode_tilt)
  #case(0)
    22.5+sin(clock*1.5*pi)*22.5
    #break
  #case(1)
    10+sin(i/lasercount*pi*2+(clock*2))*5
    #break
  #else
    0
  #end
}

\end{lstlisting}

\section{Fans und DJ}
TODO. Marco
\section{DJ-Pult}
TODO


\end{document}
